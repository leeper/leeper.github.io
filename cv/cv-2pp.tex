\documentclass[12pt]{article}
\usepackage{setspace,hyperref,xcolor}
\usepackage{fancyhdr}
\pagestyle{fancy}
\renewcommand{\headrulewidth}{0pt}
\lhead{\noindent\LARGE{Thomas J. Leeper}}

\usepackage[top=0.75in, bottom=1in, left=1.75in, right=1in, marginparwidth=1.4in]{geometry}
\setlength{\parindent}{-1.55ex}
\usepackage{libertine}
\renewcommand*\familydefault{\sfdefault}
\usepackage[T1]{fontenc}
%\usepackage[sc,osf]{mathpazo} % old serif font
\definecolor{lg}{gray}{0.5} % light-gray definition
\definecolor{mg}{gray}{0.3} % light-gray definition
\definecolor{dg}{gray}{0.0} % dark-gray definition
\color{dg}

\renewcommand{\section}[1]{\pagebreak[3]%
    \llap{\scshape\smash{\parbox[t]{\marginparwidth}{\raggedright {\color{black}#1}}}}%
    \vspace{-\baselineskip}\par}

\newcommand{\topic}[1]{\pagebreak[3]\indent {\color{lg}{\footnotesize #1 }}\\}

\newcommand{\entry}[1]{\indent {\color{lg}\guillemotright}\hspace{2pt}#1\vspace{.25em}\\}

\newcommand{\subentry}[1]{{\color{lg}-} #1\vspace{.25em}\\}

\newcommand{\hzline}[0]{\noindent\makebox[\linewidth]{\rule{\textwidth}{0.4pt}}}

\hypersetup{
    bookmarks=false,        % show bookmarks bar?
    unicode=false,          % non-Latin characters in Acrobat’s bookmarks
    pdftoolbar=true,        % show Acrobat’s toolbar?
    pdfmenubar=true,        % show Acrobat’s menu?
    pdffitwindow=false,     % window fit to page when opened
    pdfstartview={FitH},    % fits the width of the page to the window
    pdftitle={Curriculum Vitae},    % title
    pdfauthor={Thomas J. Leeper},     % author
    pdfsubject={Curriculum Vitae},   % subject of the document
    pdfkeywords={Thomas J. Leeper} {Leeper} {CV} {resume} {research}, % list of keywords
    pdfnewwindow=true,      % links in new window
    pdfborder={0 0 0}
}

\begin{document}
%\begin{minipage}[b]{0.5\linewidth}
%\end{minipage}
\begin{minipage}[b]{0.5\linewidth}
Email: \href{mailto:thosjleeper@gmail.com}{thosjleeper@gmail.com}\\
Web: \href{https://www.thomasleeper.com/}{https://www.thomasleeper.com/}\\
GitHub: \href{https://github.com/leeper}{https://github.com/leeper}\\
ORCID: \href{http://orcid.org/0000-0003-4097-6326}{0000-0003-4097-6326}
\end{minipage}\\

\vspace{-1em}
\hzline

\section{Summary of Qualifications}

\entry{PhD training plus six years academic experience producing quantitative social science research \& teaching quantitative research methods at all levels}
\entry{Recognized expertise in survey-experimental methods \& computational social science}
\entry{Prolific developer of R software and ``trusted user'' (>30,000 rep) on StackOverflow}
\hzline

\section{Relevant Professional Experience}
\entry{\textbf{Research Scientist}, \textit{Facebook Core Data Science} (Sep 2018 -- )}
\entry{\textbf{Assistant/Associate Professor}, \textit{London School of Economics} (Sep 2015 -- Aug 2018)}
	\subentry{Conducted original quantitative social science research in collaboration with various partners and industry vendors in the United States, United Kingdom, and Denmark}
	\subentry{Taught courses at undergraduate and postgraduate levels on research design, econometrics, public opinion research, and survey-experimental methods}
	\subentry{Prepared numerous written articles and presentations for a variety of audiences including academics, journalists, and the general public}
\entry{\textbf{Postdoc}, \textit{Department of Political Science, Aarhus University} (Jan 2013 -- Aug 2015)}
	\subentry{Conducted original survey research in the United States and Denmark}
	\subentry{Presented original research extensively at international professional conferences in the North America and Europe}
	\subentry{Taught courses at postgraduate level on research design, statistical analysis (regression, panel data, multilevel modelling), experimental design, and survey methodology}
\entry{\textbf{Graduate Fellow}, \textit{Institute for Policy Research, Northwestern University} (Jan 2009 -- Dec 2012)}
	\subentry{Managed Northwestern Political Research Laboratory subject pool}
	\subentry{Conducted original research using innovative survey and laboratory experimental methods to study media exposure and public opinion dynamics}
%\entry{\textbf{Research Assistant}, \textit{National Youth Leadership Council} (Jan -- Aug 2008)}
%	\subentry{Contributed quantitative research support to a team of service-learning researchers}
\hzline

\section{Education}
\entry{\textbf{PhD and MA, Political Science}, \textit{Northwestern University} (Dec 2012)}
\entry{\textbf{B.A. {\em magna cum laude}, Political Science}, \textit{University of Minnesota} (Dec 2007)}
	\subentry{Early election, Phi Beta Kappa (2007); Dean's List (2005--07)}
\hzline

\section{Skills}
\entry{Software: R, Stata, SQL/Presto, Python, LaTeX, MS Office; familiarity with SPSS, SAS}
	\subentry{Developer of more than 30 R packages}
\entry{Languages: English (native), Spanish (intermediate), Danish (beginner)}
\hzline

\clearpage

\hzline

\section{Selected\\Peer-Reviewed\\Publications}

	\entry{Alexander Coppock, Thomas J. Leeper, and Kevin J. Mullinix. ``Generalizability of Heterogeneous Treatment Effect Estimates Across Samples.'' \textit{Proceeding of the National Academy of Sciences}: In press.}
	\entry{Thomas J. Leeper, and Joshua Robison. ``More Important, but for What Exactly? The Insignificant Role of Subjective Issue Importance in Vote Decisions.'' \textit{Political Behavior}: In press.}
	\entry{Thomas J. Leeper. 2018. ``Am I a Methodologist? Asking for a Friend.'' \textit{PS: Political Science \& Politics} 51(3): 602--606.}
	%\entry{Joshua Robison, Thomas J. Leeper, and James N. Druckman. 2018. ``Do Heterogeneous Social Networks Undermine Attitude Strength?'' \textit{Political Psychology} 39(2): 479--94.\\ \href{http://doi.org/10.1111/pops.12374}{doi:10.1111/pops.12374}}
    \entry{Thomas J. Leeper. 2017. ``How Does Treatment Self-Selection Affect Inferences About Political Communication?'' \textit{Journal of Experimental Political Science} 4(1): 21--33.}	
	%\entry{Thomas J. Leeper. ``Crowdsourced Data Preprocessing with R and Amazon Mechanical Turk.'' \textit{R Journal} 8(1): 276--288. \href{https://journal.r-project.org/archive/2016-1/leeper.pdf}{https://journal.r-project.org/archive/2016-1/leeper.pdf}}
	\entry{Kevin J. Mullinix, Thomas J. Leeper, James N. Druckman, and Jeremy Freese. 2015. ``The Generalizability of Survey Experiments.'' {\em Journal of Experimental Political Science} 2(2): 109--138.}
	%\entry{Thomas J. Leeper. 2014. ``Cognitive Style and the Survey Response.'' {\em Public Opinion Quarterly} 78(4): 974--983.}
	%\entry{Thomas J. Leeper. 2014. ``Archiving Reproducible Research with R and Dataverse.'' {\em The R Journal} 6(1): 151--158. \href{http://journal.r-project.org/archive/2014-1/leeper.pdf}{http://journal.r-project.org/archive/2014-1/leeper.pdf}.}
	\entry{Thomas J. Leeper. 2014. ``The Informational Basis for Mass Polarization.'' {\em Public Opinion Quarterly} 78(1): 27--46.}
	%\entry{Thomas J. Leeper, and Rune Slothuus. 2014. ``Political Parties, Motivated Reasoning, and Public Opinion Formation.'' {\em Advances in Political Psychology} 35 (Supplement 1): 129--156. \href{http://dx.doi.org/10.1111/pops.12164}{doi:10.1111/pops.12164}}
	%\entry{Toby Bolsen, Thomas J. Leeper, and Matthew Shapiro. 2014. ``Doing What Others Do: Norms, Science, and Collective Action on Global Warming.'' {\em American Politics Research} 42(1): 65--89. \href{http://dx.doi.org/10.1177/1532673X13484173}{doi:10.1177/1532673X13484173}}
	%\entry{Toby Bolsen, and Thomas J. Leeper. 2013. ``Self-Interest and Attention to News among Issue Publics.'' {\em Political Communication} 30(3): 329--348.\\ \href{http://dx.doi.org/10.1080/10584609.2012.737428}{doi:10.1080/10584609.2012.737428}}
	\entry{James N. Druckman, and Thomas J. Leeper. 2012. ``Learning More from Political Communication Experiments: Pretreatment and Its Effects.'' {\em American Journal of Political Science} 56(4): 875--896.}
	%\entry{James N. Druckman, and Thomas J. Leeper. 2012. ``Is Public Opinion Stable? Resolving the Micro-Macro Disconnect in Studies of Public Opinion.'' {\em Daedalus} 141(4): 50--68. \href{http://dx.doi.org/10.1162/DAED\_a\_00173}{doi:10.1162/DAED\_a\_00173}}
	\entry{James N. Druckman, Jordan Fein, and Thomas J. Leeper. 2012. ``A Source of Bias in Public Opinion Stability.'' {\em American Political Science Review} 106(2): 430--454.}
\hzline

\section{Other Relevant Experience}
\entry{Editorial experience: Editorial board member, \textit{Political Behavior}, \textit{Political Communication}, \textit{Journal of Open Source Software}; Associate-PI, Time-Sharing Experiments in the Social Sciences; Peer reviewer for over 30 academic journals}
\entry{Professional awards: APSA Elections, Public Opinion, and Voting Behavior Section (2015); APSA Experimental Research Section (2015); APSA Political Psychology Section (2014); APSA Elections, Public Opinion, and Voting Behavior Section (2012); APSA Political Communication Section (2012); Honorable Mention, National Science Foundation Graduate Research Fellowship (2009)}
\entry{Recent Presentations: 2018: Durham University, University of Oxford, University of Washington; 2017: University of Konstanz, University of Vienna, ETH Zurich, NTNU Norway, European University Institute Florence; 2016: University of Essex, University of National Research University Higher School of Economics, The Ohio State University, Universitat Pompeu Fabra, Institute for Advanced Study-Toulouse; 2015: Nuffield College--Oxford, Louisiana State University, McGill University, University of Copenhagen, 2013: University of Southern Denmark; 2012: University of Chicago}
\entry{Grant Funding: Secured institutional, national, and international research funding from LSE, Northwestern University, the Danish Council for Independent Research, US National Science Foundation, and UK Economic and Social Research Council}
\hzline

\end{document}
